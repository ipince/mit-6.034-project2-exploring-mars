\section{State Representation}

The state representation used is very simple. It consists of a list of assertions that can hold all the
information about the state. The possible assertions are the following:

\begin{description}
	\item[\texttt{- (action <action>):}] Indicates the action that led to this state.
	\item[\texttt{- (at (x y)):}] Indicates the current plane location.
	\item[\texttt{- (time t):}] Indicates the time at which the rover reached this state.
	\item[\texttt{- (battery b):}] Indicates the current battery charge of the rover.
	\item[\texttt{- (need-sample (x y)):}] Indicates that the rover needs a sample from location \texttt{(x y)}.
	\item[\texttt{- (have-sample (x y)):}] Indicates that the rover holds a sample from location \texttt{(x y)}.
	\item[\texttt{- (need-communicate p):}] Indicates that the rover needs to communicate in period \texttt{p} (day).
	\item[\texttt{- (have-communicated p):}] Indicates that the rover has communicated in period \texttt{p}.
\end{description}

The goal state can then be represented as a list of \texttt{have} assertions, together with an \texttt{at}
assertion to indicate the goal location, which should be the lander, located at (0,0). The goal state should
not care about the rover's battery, the last action the rover took, nor the time at which the rover ended
the mission.

Now, if the \texttt{action}, \texttt{time}, \texttt{battery}, and \texttt{need} assertions are not needed to
encode the goal state, then why should we keep track of them in the state representation? It turns out that
this information is needed for a variety of purposes within our search framework. The action is needed to
reconstruct the search path; the time is needed to determine the outcome of certain actions (e.g., to recharge and
communicate, we need to know whether it is daylight and whether we are within a communication window,
respectively), to calculate heuristics, and to calculate the time-cost of the state; the battery is needed for
the same reasons; the need assertions are needed to determine whether the robot cares about performing some
actions (e.g., we don't want the rover to drill unless it needs to, because if it had the choice, the branching
factor of the search tree would be huge and the search would take ages).

Lastly, notice that with this representation, the goal state must specify on what communication periods
must the rover communicate, but this information may not be known a priori. To go around this issue, we just
plan missions as if the rover did not need to communicate, and if the optimal plan ends up requiring
communication to be in accordance with the problem definition, then we add the corresponding
\texttt{need-communicate} assertions and replan.