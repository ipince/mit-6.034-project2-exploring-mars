\section{Search Method}

The search method used is A* with a consistent and admissible heuristic that is described in detail later.
A* was chosen because of its speed
in finding optimal paths and because the given problem allows the creation of good heuristics that can
greatly speed the search if using a heuristic-based approach. Another approach could have been to
use best-first search, but then the information about the actual cost of each state would have been lost.
Since the heuristics we came up with were not too accurate in terms of estimating the actual cost to get to
the goal, losing information about the actual cost would have made the search much more inefficient.

Since the heuristic used is both consistent and admissible, then it made no sense not to use an expanded list.
Indeed, not using an expanded list resulted in endless (extremely long) searches, so there it was not too
difficult to decide against not using one.

	\subsection{Heuristic}
	
	The heuristic used is made up of four components: distance, battery, communication, and	waiting. Each of these
	components tries to estimate the time that the rover will spend performing the corresponding actions. The
	global heuristic just adds up all these estimates.
	
	The components were designed in such a way that estimated remaining time was never accounted for more than once
	and such that all of them were underestimates of the actual cost. This ensures that the sum of the estimates
	will be below the actual cost, so the heuristic is admissible. Also, since each of these estimates depends on
	the current and goal nodes and the estimation function used is increasing as the current state approaches the
	goal, then the resulting heuristic is also consistent.
	
	Detailed descriptions of each of the components are found in the following sections.
	
		\subsubsection{Distance}
		
		The distance component estimates the remaining time the rover will spend driving. It does so by first estimating
		the maximum distance it absolutely \emph{needs} to cover. This is calculated in the following way: for each of
		the remaining drilling locations, the manhattan distance from the current location to the drilling location
		and from the drilling location to the goal location is calculated. If there are no remaining drilling locations,
		then only the manhattan distance from the current location to the goal location is considered. The estimate we
		are looking for is the maximum of these distances. Once this is calculated, we estimate the time cost of
		driving by multiplying the distance estimate with the minimum cost of driving.
						
		It would be nice to use a better estimate than this one. A couple alternatives were considered. First, instead
		of considering the maximum distance from the rover's current location to a drilling site and from the drilling
		site to the goal, we could have used a path-based approach. We quickly decided against this since this is
		just a formulation of the travelling salesman problem. Solving that problem at every search node to calculate
		the heuristic value would be too wasteful and slow.
		
		Second, we thought of using the actual cost of each path instead of the product of the minimum cost and the
		total distance. The problem here is that this is not well-defined; moving from location \textit{A} to location
		\textit{B} may be done through different paths that result in different costs. To choose which path to use, we
		could just take the shortest path, but again, this would involve solving another complicated problem that
		would take too long. Another option was to choose any random path, but this means that the heuristic function
		would no longer be admissible, so this option was rejected too.
		
		Now, notice that the estimate, in the way it is implemented, works best when the time cost of driving in the
		map is close to the minimum cost. If this is not the case, our estimate could be way below the actual cost and
		the heuristic is not that useful.
		
		\subsubsection{Battery}
		
		The battery component estimates the remaining time the rover will spend recharging. It does so by calculating
		the amount of battery units needed to complete the mission in terms of battery spent moving to the farthest
		drilling site and back to the goal, battery spent drilling, and battery spent communicating. This sets a bound
		on the minimum number of battery units necessary. If we subtract the current battery charge from this, then we
		get an estimate for the minimum number of extra battery units that the rover will need for the mission. Since it
		takes	one hour to charge on unit of battery, this also corresponds to the minimum amount of time the rover will
		have to spend charging.
		
		\subsubsection{Communication}
		
		The communication component estimates the amount of time the rover will spend communicating. It does so by
		calculating the amount of communications are still needed to complete the mission and multiplying that by the
		communication time-cost.
		
		As was mentioned in Section \ref{actions}, if the rover has failed to communicate when it needed to, then an
		insanely large number (100,000 hours) is given as the estimate.
		
		\subsubsection{Waiting}
		
		Finally, the waiting component estimates the time that the rover will have to waste while waiting in the
		remainder of the mission. To keep things very simple, it only estimates a non-zero waiting time if the
		current time lies during the night. If so, then it calculates the ratio of the current amount of battery
		to the minimum amount of battery needed to do anything that is not waiting. This determines the maximum
		number of hours the rover can spend while not waiting. Thus, subtracting this number from the number of
		hours left until daybreak yields an estimate for the minimum amount of hours that the rover will spend
		waiting. Notice that we only consider the minimum amount of battery spend on a non-waiting action. Why?
		Waiting incurs a battery cost of 0, but if the rover decides to wait, then it will still waste time by
		waiting! Also, we make things simple by only considering times that are during the night. At these times,
		the rover cannot recharge nor communicate without waiting, so the minimum non-waiting battery cost is
		simply given by the minimum battery cost of driving.
		
		This particular component of the heuristic could have been improved. For instance, if the rover has 4 units
		of battery at 4 PM and the minimum battery cost of driving, drilling, and communicating is 1, then we can
		conclude that it will have to waste at least 10 hours waiting. This is not incorporated into the waiting
		component and could potentially make a big difference. It was not incorporated since it is hard to
		determine the waiting outcomes at any arbitrary point of the day.