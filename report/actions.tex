\section{Action Representations} \label{actions}

The action representations (or procedures) were also designed with a minimalistic approach. We wanted to
have the least amount of actions possible at any given state, so that the branching factor of the search
tree would be as small as possible. Notice that it is not enough to have few action procedures. Each of
these is a function of the current state and a set of arguments. The number of arguments and their
domains also affect the branching factor, which is why we tried to make each action procedure take as few
arguments as possible and make their domains as retrictive as possible.

Following this approach, we ended up having one action procedure per physical action, except for waiting.
This is another key point. Since we are optimizing for time, there is no point in waiting unless the rover
is forced to. This can happen if it needs to recharge during the night or if it is waiting for the next
communication window. Thus, we eliminated the need of having an action procedure for waiting by
incorporating the waiting effects into the recharge and communicate procedures.

With this in mind, we designed and implemented the following set of actions:

\begin{description}
	\item[\texttt{- (go d):}] The go action enables the rover to drive from its current location (kept in
		the state) to a neighboring location. The direction argument \texttt{d} can assume the values
		\texttt{x-}, \texttt{x+}, \texttt{y-}, or \texttt{y+}, and specifies which ending location to go to.
		This the only action that takes an argument.
	\item[\texttt{- (drill):}] The drilling action enables the rover to take a sample from its current
		location. The rover must be in need of a sample from that location in order to be able to take one.
	\item[\texttt{- (recharge):}] The recharge action completely recharges the rover's battery. This
		action incorporates all the details involving the recharging specification. For instance, if recharge
		is performed during the night, then the rover will wait until daylight to recharge, so the resulting
		state could possibly be over 10 hours in the future.
	\item[\texttt{- (communicate):}] The communicate action makes the rover communicate only if it
		actually \emph{needs} to communicate in the current period. If it tries to communicate while
		not being inside a communication window, then it will wait until the next available window and
		communicate then.
\end{description}

As a remark, notice that at each state, there is at most 7 things the rover could do: move in one of four
possible directions, drill, recharge (and wait as needed in accordance with the problem definition), or
communicate (and wait as needed).

There are some requirements that have not been accounted for with the current state and action
representations. For instance, if a rover needs to communicate on period 3, but the communication window
corresponding to that period has already passed, then it is impossible for the rover to complete the mission
successfully. Instead of performing checks for this in each action and not allowing the rover to proceed
once it has reached this state, we chose to keep things simple and account for this within our heuristic
function. Thus, if the rover has failed to communicate, our heuristic estimate will be some insanely large
number, which will make the search algorithm avoid such path.