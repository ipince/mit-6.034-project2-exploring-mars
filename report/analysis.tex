\section{Performance}

We performed tests in settings with different maps, costs, initial, and goal states. In this section, we only
illustrate three of them. The following are descriptions of the three settings:

\begin{description}
	\item[Test 1] Uniform map with minimum time and battery driving costs of 1 and 2, respectively; samples
		needed from (1 1) and (2 1); communication needed in period 0; default initial conditions (location
		(0 0), time 0, and battery 10).
	\item[Test 2] Same scenario as above, except that the minimum costs are now both 1, and there is an
		added drill location at (3 3).
	\item[Test 3] Map with minimum time and battery costs of 1, but where the rover is almost surrounded
		by high-cost terrain. It only needs to retrieve one sample, located at (3 0), which is also the goal
		location. However, the direct path to (3 0) is blocked by high-cost terrain. In spite of this, it
		is still better to go through the high-cost terrain to finish the mission (ticky map!).
\end{description}

For each setting, we ran the search procedure with no heuristic, using only one component of the heuristic, and
using the complete heuristic. The results are shown in Table \ref{times}. The entries in each table correspond
to the time taken, in seconds, followed by the number of nodes that were expanded. If an entry is ``endless,''
it means that the search had not finished after over an hour.

\begin{table}
 	\caption{Performance (time, expanded) on three settings using different heuristics.}
   \centering
 	\begin{tabular}{|l|c|c|c|c|c|c|}
 		\hline
 		\textbf{Setting} & \textbf{No Heuristic} & \textbf{Distance} & \textbf{Battery} & \textbf{Communication} & \textbf{Waiting} & \textbf{All}\\ \hline
 		Test 1           & 637.9, 5,501  & 17.6, 644   & 17.4, 650    & 236.5, 3,233 & 663.1, 5,501 & 1.5, 107      \\ \hline
 		Test 2           & endless       & endless     & endless      & endless      & endless      & 326.5, 2,956  \\ \hline
 		Test 3           & endless       & 30.1, 1,073 & 570.8, 4,341 & endless      & endless      & 184.2, 2,677  \\ \hline
 	\end{tabular}
 	\label{times}
\end{table}

From the table, we can see that the distance and battery components of the heuristic do a much better job than
the other two. It is impressive how the search time dropped from 637 seconds to 1.5 in the first test when using
no heuristic and when using the complete heuristic. It is also interesting to note that using only the waiting
component yielded a worse result than when using no heuristic at all. The number of expanded nodes and the solution
itself were the same. This means that the heuristic was not very useful and we had to pay the time cost of
calculating it many thousand times.

However, notice that using the complete heuristic is not always the best option. The last test is a good example.
In this case, the rover needs to overcome some rough terrain in order to get to its goal. When using the distance
component only, the search is biased towards the goal location, which makes it end quickly. On the other hand,
then using the battery component we get much worse results. This happens because since rough terrain incurs
big battery costs. When the rover moves to a rough terrain location, it loses a lot of battery. However,
its battery estimate is only decreased by a little, since we are only considering the minimum cost. Hence, it
appears as if the rover had lost some units of battery in vain. Because of this, the search will try to avoid this
path and go around the rough terrain. This makes the search take a lot longer, as can be seen in the table.
The complete heuristic then gives a worse result than a distance-based heuristic by itself because the battery
estimate plays a role in the complete one.