\section{Overview}

This paper tackles the problem of planning a rover's mission in Mars. The rover starts the mission at the
lander with full battery. Its objective is to retrieve samples from a set of specified map locations,
communicate with an orbiting satellite during specified communication windows, and return to the lander.
Planning a mission involves determining the sequence of actions that the rover will follow to accomplish
the mission.

The following are descriptions of the actions the rover can perform:

\begin{description}
	\item[Drive] The rover can move throughout a grid map (lattice plane) by driving to any of the 4 adjacent
		points to its current location. It can drive at any time of day. Driving incurs a time cost and a battery
		cost that may depend on the rover's current location.
	\item[Recharge] The rover may choose to recharge its battery. Once it starts recharging, it must keep
		recharging until the battery is fully charged. The rover can only charge during the day. That is, if night
		falls during recharging, the rover will wait until daybreak and resume recharging. Each hour spent
		recharging results in a battery unit. A fully charged battery has 10 battery units.
	\item[Drill] The rover can drill to take a scientific sample from its current location. This takes 1 hour and
		requires 1 unit of battery.
	\item[Communicate] While on the mission, the rover must start communication in every communication window. These
		occur every day at 10 AM Mars time (see assumptions below) and last for 2 hours. Communicating takes 15
		minutes and requires 1 unit of battery.
	\item[Wait] The rover can sit and do nothing without consuming any battery.
\end{description}

None of these actions can overlap in time. Additionally, we assume the Mars day has 24 hours, daylight hours
are from 6 AM to 6 PM, and the landing occurs at midnight, Mars time.

Given this setup, we tackled the problem from a search perspective. This paper describes the state
and action representations used to model the problem, the search approach used, and an analysis of the
time performance of the chosen method. In general, a minimalistic approach was used to model state and
actions, in order to increase the performance of the search. The search method used was A* with a heuristic
that incorporated four different estimates: time spent moving, recharging, communicating, and waiting.